\begin{abstract}
Performing operations on the traffic of high-speed networks face the typical problem of 
volume, velocity and variety. Network algorithmics is a field developed to deal 
with these problems. Of importance to network algorithmics is the inputs/workloads into
the solutions developed by the field are highly variable, and so is often modeled as
a stochastic process. Moreover, a chief cornerstone of network algorithmics are randomized
algorithms, which tradeoff a resource (for example, space) for a small, allowable
error. Thus, it is crucial that one has to derive probabilistic guarantees on the performance of a
solution with a worst-case workload under a performance metric (for instance, estimation error).
This is usually achieved via probability bounds such as the famous Chernoff bound, but deriving
such bounds for solutions from network algorithmics is often challenging.

The derivation of these bounds can be dramatically simplified by exploiting
a stochastic order, which is a notion of when one random variable is ``larger'' than
another. In this paper, we survey some stochastic orders and show how they can
be used to provide simple proofs of these bounds in the design of randomized
algorithms in network algorithmics.
\end{abstract}

\keywords{
Log-concave distributions, majorization, network algorithmics, stochastic ordering, supermodularity.
}