%\documentclass[onecolumn]{IEEEtran}
\documentclass{article}

\usepackage[usenames,dvipsnames]{xcolor}
%\usepackage[utf8]{inputenc} 
\usepackage{fancyhdr}
\usepackage{epsf}
\usepackage{subfigure}
\usepackage{dsfont}
\usepackage{url}
\usepackage{times}
\usepackage{ifthen}
\usepackage{verbatim}

\usepackage{bbm,amssymb,amsmath,amsfonts,amsthm,latexsym}
\usepackage{graphicx,color}
\usepackage{tikz,pgf}
\usepackage{algorithm,algorithmic}
\usepackage{graphics,url}
\usepackage{subfigure}
\usepackage{tkz-base,tkz-berge,tkz-graph,tkz-fct,tkz-euclide}

%% Needed so we can draw shapes
\usetkzobj{all}


%\usepackage{floatrow} % for changing table fontsize
%\DeclareFloatFont{normal}{\normaltext}
%\floatsetup[table]{font=normal}

%%%%%%%%% Colour effects
\definecolor{subtler}{rgb}{1,0,0.1}  % subtle shading
\def\y{\color{subtler}}              % color of changes
\def\b{\color{blue}\small}           % talking to colleagues
\def\pt{\color{black}}

% captions and subcaptions
% \usepackage[caption=false,position=bottom]{subfig}
\usepackage{caption}
\let\subcaption\relax
\let\subfloat\relax
%\usepackage{subcaption}
% \usepackage[caption=false,position=bottom]{subcaption}
% \let\subbottom\subfloat % to allow use of \subbottom instead of \subfloat
\captionsetup[figure]{labelfont={bf},textfont={it}}
\captionsetup[table]{labelfont={bf},textfont={it}}
\captionsetup[subfloat]{labelfont={bf,footnotesize},textfont={it,footnotesize},subrefformat=parens}

% hyperlinks, which caused some fof the grief above
%\let\captioncaption\caption
\usepackage[colorlinks=true,linkcolor=blue,citecolor=blue,urlcolor=blue,pdfauthor={Paul Tune and Jun Xu},pdftitle={Stochastic Orders
for Network Algorithmics}]
{hyperref}
%\let\caption\captioncaption
%\usepackage[all]{hypcap}

%%%% Formatting
\setlength{\headheight}{10mm}
\setlength{\headsep}{10mm}
\setlength{\topmargin}{-1cm}
\setlength{\textwidth}{175mm}
\setlength{\textheight}{220mm}
\setlength{\oddsidemargin}{-5mm}
\lefthyphenmin=2
\righthyphenmin=3

\renewcommand{\baselinestretch}{1.0}
\renewcommand{\textfraction}{0.1}

%%%%% Page headers
\newcommand{\titlestr}{Stochastic Orders}
\rhead[]{{\small \titlestr}}
\lhead[{\small }]{}
\cfoot{}
\rfoot[]{{\rm\thepage}}
\lfoot[{\rm\thepage}]{}
\renewcommand{\footrulewidth}{0.4pt}
\pagestyle{fancy}


%------- My common Definitions
\newcommand{\be}{\begin{equation}}
\newcommand{\ee}{\end{equation}}
\newcommand{\ben}{\begin{equation*}}
\newcommand{\een}{\end{equation*}}
\newcommand{\ba}{\begin{eqnarray}}
\newcommand{\ea}{\end{eqnarray}}
\newcommand\Var {{\rm Var}}
\def\Cov {\makebox{Cov }}
\newcommand{\Z}{Z\!\!\!Z}
\def\integer {\cal Z}
\def\real {{\cal R}}
\def\bl{\Bigl(\,}
\def\br{\,\Bigr)}

%%%%%%%%%%%%%%%%% definitions for this document
\providecommand{\keywords}[1]{\begin{center}\textbf{\textit{Index terms---}} #1 \end{center}}
\def\mus{{$\mu$s}}         % microseconds
% THEOREMS -------------------------------------------------------
\newtheorem{thm}{Theorem}%[section]
\newtheorem{cor}[thm]{Corollary}
\newtheorem{lem}[thm]{Lemma}
\newtheorem{prop}[thm]{Proposition}
\newtheorem{defn}[thm]{Definition}
\newtheorem{rem}[thm]{Remark}
\newtheorem{exm}[thm]{Example}
% MATH -----------------------------------------------------------
\newcommand{\id}[1]{\mathbf{I}_{#1}}
\newcommand{\mnull}[1]{\mathbf{0}_{#1}}
\newcommand{\vnull}[2]{\mathbf{0}_{#1 \times #2}}
\newcommand{\norm}[1]{\left\Vert#1\right\Vert}
\newcommand{\abs}[1]{\left\vert#1\right\vert}
\newcommand{\set}[1]{\left\{#1\right\}}
\newcommand{\mset}[1]{\lbrack #1\rbrack}
\newcommand{\Real}{\mathbb R}
\newcommand{\Realp}{\mathbb R_{+}}
\newcommand{\Complex}{\mathbb C}
\newcommand{\Integer}{\mathbb Z}
\newcommand{\Natural}{\mathbb N}
\newcommand{\eps}{\varepsilon}
\newcommand{\To}{\longrightarrow}
\newcommand{\BX}{\mathbf{B}(X)}
\newcommand{\I}{\boldsymbol{\mathcal{I}}_{\vth}}
\newcommand{\tr}{\mathbf{tr}}
\newcommand{\adj}{\mathrm{adj}}
\newcommand{\T}{\mathrm{T}}
\newcommand{\E}{\mathbb E}
\newcommand{\bEu}[1]{\mathbf{e}_{#1}}
\newcommand{\bin}[2]{\text{Bin}(#1,#2)}
\newcommand{\Count}{1}
\newcommand{\constr}{\mathbf{G}}
\newcommand{\ind}{\mathbbm 1}
\newcommand{\veth}{\boldsymbol{\hat \theta}}
\newcommand{\vbth}{\boldsymbol{\bar \theta}}
\newcommand{\iset}[1]{\lbrack #1 \rbrack}
\newcommand{\bxi}[1]{\mathbf{x}^{(#1)}}
\newcommand{\bXi}[1]{\mathbf{X}^{(#1)}}
\newcommand{\iomega}[1]{\omega^{(#1)}}
\newcommand{\bdi}[1]{\mathbf{d}^{(#1)}}
\newcommand{\perm}[2]{{#1}_{\lbrack #2 \rbrack}}
\newcommand{\Exp}[1]{\text{Exp}({#1})}
\newcommand{\DKL}[2]{D_{KL}(#1\,||\,#2) }
\newcommand{\meet}{\mathrel{\text{\raisebox{0.25ex}{\scalebox{0.8}{$\wedge$}}}}}
\newcommand{\join}{\mathrel{\text{\raisebox{0.25ex}{\scalebox{0.8}{$\vee$}}}}}
% ----------------------------------------------------------------
\renewcommand{\labelenumi}{(\roman{enumi})}
\renewcommand{\qedsymbol}{$\blacksquare$}
\newcounter{tempcnt}


% use "autoref" the way I want
%   autoref is nice because the text or brackets of a ref are part of link
%   see http://tex.stackexchange.com/questions/36575/autorefs-inserted-text-has-not-the-correct-case
%       http://en.wikibooks.org/wiki/LaTeX/Labels_and_Cross-referencing
%       http://www.tug.org/applications/hyperref/manual.html#TBL-23
\def\algorithmautorefname{Algorithm}
\def\sectionautorefname{Section}
\def\subsectionautorefname{Section}
\def\subfigureautorefname{Figure}
\def\theoremautorefname{Theorem}
% from http://tex.stackexchange.com/questions/52410/how-to-use-the-command-autoref-to-implement-the-same-effect-when-use-the-comman
\def\equationautorefname~#1\null{%
  (#1)\null
}

\def\gap{\vspace{10pt}\noindent}
\def\naive{na\"{\i}ve\ }
\def\onen{\mathbf{1}_n}
\def\range{\text{range}}
\def\Cov{\text{Cov}}

\def\vth{\boldsymbol\theta}
\def\sech{\text{sech}}

\newcommand{\etal}{{\em et al.}~}
\newcommand{\ie}{{\em i.e.,} }
\newcommand{\eg}{{\em e.g.,} }
\newcommand{\etc}{{\em etc.,} }

% defs for this paper
\newcommand{\Nat}{\mathbb N}
\def\bJ{\mathbf{J}_{\vth}}
\def\bc{\mathbf{c}_{\vth}}
\def\bW{\mathbf{W}}
\def\bG{\mathbf{G}}
\def\bA{\mathbf{A}}
\def\bC{\mathbf{C}}
\def\bD{\mathbf{D}}
\def\bE{\mathbf{E}}
\def\bH{\mathbf{H}_{\vth}}
\def\bM{\mathbf{M}}
\def\bN{\mathbf{N}}
\def\bP{\mathbf{P}}
\def\bW{\mathbf{W}}
\def\bU{\mathbf{U}}
\def\bQ{\mathbf{Q}}
\def\bV{\mathbf{V}}
\def\bF{\mathbf{F}}
\def\bR{\mathbf{R}}
\def\bX{\mathbf{X}}
\def\bXs{\mathbf{\bar X}}
\def\bY{\mathbf{Y}}
\def\bZ{\mathbf{Z}}
\def\bB{\mathbf{B}}
\def\bTheta{\boldsymbol{\Theta}}
\def\vec{\mathrm{vec}}
\def\bee{\mathbf{e}}
\def\ba{\mathbf{a}}
\def\bb{\mathbf{b}}
\def\bq{\mathbf{q}}
\def\bv{\mathbf{v}}
\def\bx{\mathbf{x}}
\def\bbx{\mathbf{\bar{x}}}
\def\by{\mathbf{y}}
\def\bby{\mathbf{\bar{y}}}
\def\bu{\mathbf{u}}
\def\bz{\mathbf{z}}
\def\bw{\mathbf{w}}
\def\bb{\mathbf{b}}
\def\bp{\mathbf{p}}
\def\bd{\mathbf{d}}
\def\cc{\mathbf{\bar c}}
\def\bI{\mathbf{\tilde I}}

\def\tbA{\boldsymbol{A}}
\def\tbD{\boldsymbol{D}}
\def\tbX{\boldsymbol{X}}
\def\tbY{\boldsymbol{Y}}

\def\cA{\mathcal{A}}
\def\cC{\mathcal{C}}
\def\cD{\mathcal{D}}
\def\cG{\mathcal{G}}
\def\cH{\mathcal{H}}
\def\cL{\mathcal{L}}
\def\cS{\mathcal{S}}
\def\cI{\mathcal{I}}
\def\cE{\mathcal{E}}
\def\cM{\mathcal{M}}
\def\cJ{\mathcal{J}}
\def\cT{\mathcal{T}}
\def\cX{\mathcal{X}}

\def\iY{\boldsymbol{Y}}

\def\Spd{\mathbb{S}^{n}_{++}}
\def\Spsd{\mathbb{S}^{n}_{+}}
\def\diag{\text{diag}}
\def\rank{\text{rank}}

\def\SD{\text{StdDev}}
\def\Var{\text{Var}}
\def\Cov{\text{Cov}}
\def\ce{\mathbf{c}_e}
\def\oneinf{\mathbf{1}_{\infty}}
\def\ones{\mathbf{1}}
\def\Fourier{\boldsymbol{\mathcal{F}}}

\def\CovMtx	{\boldsymbol{\Sigma}_{\vth}}
\def\CovMtxSeq	{\boldsymbol{\Sigma}_{\hat{\vth}^{(N)}}}
\def\bSigma {\boldsymbol{\Sigma}}

%%%%%% Orders
\def\lecx{\le_{\text{cx}}}
\def\leicx{\le_{\text{icx}}}
\def\lst{\le_{\text{st}}}
\def\lsm{\le_{\text{sm}}}
\def\lmaj{\le_{\text{M}}}


%----- Figure to file mappings
%\def\homedir{.}
%\def\figuresdir{\homedir}
\graphicspath{{.}
{Figures/}
}


\def\oneup{86mm}
\def\twoup{80mm}
\def\threeup{58mm}


\begin{document}

\title{Preparing for the Worst: Applications of Stochastic Orders in Network Algorithmics}
\author{Paul Tune and Jun (Jim) Xu\thanks{The first author is with the School of Mathematical Sciences, The University of Adelaide, 
Australia, and the second is with the College of Computing, Georgia Institute of Technology (Email: paul.tune@adelaide.edu.au,
jun.xu@cc.gatech.edu).}}
\date{}
\maketitle

\begin{abstract}
Performing operations on the traffic of high-speed networks face the typical problem of 
volume, velocity and variety. Network algorithmics is a field developed to deal 
with these problems. Of importance to network algorithmics is the inputs/workloads into
the solutions developed by the field are highly variable, and so is often modeled as
a stochastic process. Moreover, a chief cornerstone of network algorithmics are randomized
algorithms, which tradeoff a resource (for example, space) for a small, allowable
error. Thus, it is crucial that one has to derive probabilistic guarantees on the performance of a
solution with a worst-case workload under a performance metric (for instance, estimation error).
This is usually achieved via probability bounds such as the famous Chernoff bound, but deriving
such bounds for solutions from network algorithmics is often challenging.

The derivation of these bounds can be dramatically simplified by exploiting
a stochastic order, which is a notion of when one random variable is ``larger'' than
another. In this paper, we survey some stochastic orders and show how they can
be used to provide simple proofs of these bounds in the design of randomized
algorithms in network algorithmics.
\end{abstract}

\keywords{
Log-concave distributions, majorization, network algorithmics, stochastic ordering, supermodularity.
}

\section{Introduction}

Over the past two decades, the field of \emph{network algorithmics}~\cite{Varghese04Algorithmics}
has emerged as a rich area of research. The types of problems that network
algorithmics address include, but are not limited to, packet classification,
queue management, traffic shaping, switching and routing, network measurement,
and data streaming analysis. Many of such solutions have been successfully
deployed in commercial networks, including solutions for Internet routers,
security apparatus, and measurement devices. 

In general, network operators
would like solutions that are robust under a wide variety of, \emph{often
unforeseen}, operating conditions. However, the performance of a network appliance
design or the accuracy of an estimation method is often dependent on workload
uncertainties that are beyond the control of the operator. Unfortunately,
applicable mathematics for the rigorous analysis of the worst-case stochastic behaviors of
network algorithmics solutions under arbitrary workloads is largely lacking.

Understanding how a solution would behave in the worst-case, not just in
the typical case, is important for two reasons. First, with suitable mathematics
to characterize worst-case workloads, we can design solutions that will work
well under any operating conditions, including those in which an adversary is
trying to break the system, or under unexpected changes in the usage pattern.
Second, more often than not, we have, surprisingly, found in our past efforts
that delivering solutions that can guarantee high performance under the worst-case
conditions cost only slightly more than designs that don't. However, coming up
with such solutions hinges on our ability to understand the characteristics
of the worst-case scenarios so that we can design around them.

Large deviation theory~\cite{Dembo98LDV} on $\Real$ is concerned with the probability that
the sum of some random variables $S := \sum_{i=1}^n X_i$ will exceed a given
threshold $x$, which in our contexts may correspond to processing or network capacity,
resource constraint, or tolerable error bound. In worst-case large deviation problems, 
these random variables $X_1, X_2, \cdots, X_n$
are the functions of some large parameter vector $\bq := \lbrack q_1, q_2, \cdots, q_m\rbrack$,
and we wish to find the worst-case probability tail bounds of
$S(\bq) := \sum_{i=1}^n X_i(\bq)$ under all possible parameter
settings $\bq \in Q$. In other words, we would like to compute
$\max_{\bq\in Q} \Pr[S(\bq) \ge x]$.

Establishing such worst-case tail bounds is important not only for network
security applications where an adversary is in full or partial control of
this vector, but also for non-security applications where we would like to
know the worst-case system performance under all operating conditions or workloads.
Obtaining such worst-case bounds is very difficult because the parameter
space $Q$ is typically gigantic. Although establishing tail bounds is
often straightforward through Chernoff bounding
techniques when a particular parameter setting $\bq$ is given,
such bounds typically cannot be expressed as a closed form function of $\bq$,
rendering conventional optimization techniques powerless for maximizing $\Pr[S(\bq) \ge x]$.
Without worst-case large deviation machineries that have been or are being developed, the only
conceivable option would be to enumerate all parameter settings $\bq\in Q$, but repeating tail
bound analysis over the entire parameter space $Q$ is usually computationally prohibitive.

Whether determining the worst case input or bounding the
behavior of randomized algorithms, \textit{stochastic orders} can help in understanding the worst-case behavior of proposed
solutions in network algorithmics. The motivation of this paper is to provide an overview to researchers in
network algorithmics on stochastic orders.

A stochastic order basically defines when a random variable is ``larger'' than another. The precise definition
of ``larger'' than will depend on the order. 
There are many types of stochastic orders that can be defined on random variables. Two chief references are
M\"{u}ller and Stoyan \cite{Muller02Risk}, and Shaked and Shantikumar \cite{Shaked07Sorders}. Of importance
to us are majorization and Schur convexity, supermodular and convex orderings. We will also briefly mention about negatively associated
random variables and how they relate to network algorithmics.

In this paper, we will summarize some useful classifications of distributions and stochastic orders to aid
in the design of randomized algorithms in network algorithmics. 
In particular, we cover:
\begin{enumerate}
\item majorization and Schur convexity, 
\item negative association,
\item convex ordering and supermodularity, and
\item log-concave distributions.
\end{enumerate}
We show how these can be applied to various applications in network algorithmics. Most of the 
results shown here can be found in literature, and so we endeavor to reference the original 
paper where the result originated from. 





%High-speed networks pose a challenge to performing
%operations, such as measurement, on network traffic, due to three main
%factors (also encountered in Big Data applications): volume, velocity and variety. Volume
%and velocity are self-evident: it is common for today's backbone networks to run around 
%100 Gbps, and this speed is likely to be surpassed in the near future. Variety appears in 
%the different forms of traffic an operator would like to measure, for instance, traffic
%flows of particular applications such as Voice-over-IP (VoIP), or to flag anomalies
%in traffic.
%
%On a 100 Gbps link, several Terabytes of data are being transmitted per hour.
%Clearly, the \naive solution of performing operations by first storing all traffic is prohibitively expensive, both
%from a computational and financial standpoint. Network algorithmics, coined by 
%Varghese \cite{Varghese04Algorithmics}, is a field that arose to address these challenges. 
%Some problems network algorithmics has addressed include, but not limited to, packet classification
%(such as deep packet inspection), queue management, traffic shaping, switching and routing, network
%measurement, and data streaming analysis. The solutions from network algorithmics have been 
%successfully deployed in commercial networks, such as new switching architectures, security apparatus, and
%measurement devices. As network speeds increase, couple with increasing traffic demand, network
%algorithmics will be even more prominent in the near future.
%
%As the inputs, \ie traffic streams, into these solutions are 
%often arbitrary, an important issue in network algorithmics is to deal with the worst-case scenario the solutions
%could face. Even though the solution itself has a deterministic behavior, how it behaves will be largely
%determined by the stochastic behavior of the inputs. For instance, in switching, a scheduling policy with 
%deterministic behavior is encouraged because deterministic behavior simplifies the design of the 
%on-chip controller that implements the scheduling policy. The simple yet effective deterministic 
%round-robin policy (with priority queues) of the well-known iSLIP \cite{McKeown99iSlip} 
%scheduling policy explains its success in widespread implementation on Cisco's routers. However,
%analysis of its worst-case behavior is determined by its interaction with specific traffic workloads.
%
%Moreover, many solutions in network algorithmics rely on \textit{randomness} itself. 
%The dual requirements of dealing with high speeds constrained by a small amount of memory necessitates
%the use of compact summaries with \textit{randomized} logic in network measurement. 
%As the name suggests, these randomized algorithms 
%utilizes randomness in its operations. One famous example for testing the membership of an item in a set is the Bloom filter
%\cite{Bloom70Filter}. The Bloom filter enables the compact representation of a set for querying. 
%Suppose a set of items are stored in the filter. If an item is in the set of items, querying for 
%the item will always return a ``Yes''. However, if we have a query for an item \textbf{not} found in the
%set, there is a small probability that the answer will be ``Yes''. This error is known as a false positive.  
%In this sense, the Bloom filter trades off some storage space with a small probability of \textit{error}. 
%
%Essentially, this is the crux of (Monte Carlo) randomized algorithms: ensure a bounded resource (for example,
%space), but trade this off with some error on its operations \cite{MitzenmacherProb05}. 
%Randomized algorithms in network algorithmics require probability bounds on these errors (or ``bad'' events) occurring. 
%
%In both cases above, \textit{stochastic orders} can help in understanding the worst-case behavior of proposed
%solutions in network algorithmics. The motivation of this paper is to provide an overview to researchers in
%network algorithmics on stochastic orders.
%
%A stochastic order basically defines when a random variable is ``larger'' than another. The precise definition
%of ``larger'' than will depend on the order. 
%There are many types of stochastic orders that can be defined on random variables. Two chief references are
%M\"{u}ller and Stoyan \cite{Muller02Risk}, and Shaked and Shantikumar \cite{Shaked07Sorders}. Of importance
%to us are supermodular and convex orderings. We will also briefly mention about negatively associated
%random variables and how that relates to network algorithmics.
%
%In this paper, we will summarize some useful classifications of distributions and stochastic orders to aid
%in the design of randomized algorithms in network algorithmics. 
%In particular, we cover:
%\begin{enumerate}
%\item majorization and Schur convexity, 
%\item negative association,
%\item convex ordering and supermodularity, and
%\item log-concave distributions.
%\end{enumerate}
%We show how these can be applied to various applications in network algorithmics. Most of the 
%results shown here can be found in literature, and so we endeavor to reference the original 
%paper where the result originated from. 


\section{Background and Motivation}

As mentioned in the introduction, the study of random processes are central in network algorithmics. 
Common mathematical tools to bound the worst-case behavior includes the Markov's inequality,
Chebyshev's inequality, the Chernoff bound and 
Azuma-Hoeffding inequality \cite{Azuma67Martingale,Hoeffding63Bounded}. Overviews of these
tools (and many others) can be found in \cite{Alon04ProbMethod,MitzenmacherProb05,Raghavan95RandAlgo}.

Why do we care about finding strong, refined worst-case probability bounds? Or equivalently, can one just
make do with coarse bound?

The question can only be answered in context: if the problem can tolerate a wide margin of
``bad'' events, then a refined worst-case probability bound is unnecessary. More often than not, however,
due to the high-speed, large data volume environment, stronger bounds are a necessity.

Consider the issue of packets arriving out-of-order after traversing a switch. TCP-based applications
are sensitive to this packet reordering. A TCP-based application will drop the set of packets it has received thus far, and
issue a retransmission to the sender, causing wastage of bandwidth. Suppose as a designer of a switch, at 
worst, we know that packets will arrive out-of-order with probability $1/1000$. Then, roughly 1 out of a 
1000 packets will be out-of-order. Suppose each packet is has size 100 bytes, then on a 100 Gbps link,
it takes 8 ns to transmit a single packet. Then, we expect about 1 packet to be out-of-order every 8 
$\mu$s. This may not be acceptable for some applications, so stronger bounds are required.

The lesson here is that the volume of data processed by solutions from network algorithmics are massive,
and certainly large enough for rare events to be seen in a given sample set. In the rest of the paper, we
discuss some useful stochastic orderings and properties of random variables that help with the development
of tighter worst-case bounds. 



\section{Majorization and Schur Convexity}

The concept of \textit{majorization}, essentially a partial order on the distribution of random variables defined
below, has seen widespread applications in various fields. The \textit{de facto} reference for majorization is
Marshall and Olkin's book \cite{Marshall79Majorization}. We shall briefly discuss majorization here.

Let us define majorization and \textit{Schur convex functions}.
The following definitions can be found in \cite{Marshall79Majorization}.
\begin{defn}
For any $n$-dimensional vectors $\bx$ and $\by$, let $x_{[1]} \ge \cdots \ge x_{[n]}$ and $y_{[1]} \ge \cdots \ge y_{[n]}$
denote the components of $\bx$ and $\by$ in non-increasing order respectively. We say $\bx$ is majorized by $\by$,
denoted by $\bx \lmaj \by$ if $\sum_{i=1}^k x_{[i]} \le \sum_{i=1}^k y_{[i]}$ for $k=1,2,\cdots,n-1$ and  
$\sum_{i=1}^n x_{[i]} = \sum_{i=1}^n y_{[i]}$.
\label{def:majorize}
\end{defn}

\begin{defn}
A function $f : \Real^n \to \Real$ is Schur convex (Schur concave) if $\bx \lmaj \by$ implies $f(\bx) \le f(\by)$ ($f(\bx) \ge f(\by)$).
\label{def:schur_convex}
\end{defn}

In statistics parlance, $x_{[i]}$ denotes the $i$-th \textit{order statistic} of a sample value. Majorization equivalently
says that the order statistic of one sample set dominates another. 

In \cite{Zhao10GlobalIceberg}, majorization was used in conjunction with another stochastic order to derive
a tail bound on the overestimation error of a \textit{icebergs} or heavy-hitters, \ie objects with a large count/size,
on distributed streams of data. We shall discuss this in more detail in \autoref{sec:supermodular}.

We look at one interesting result.
Let $X_1,X_2,\cdots, X_n$ be Bernoulli random variables with success probabilities $p_1,p_2,\cdots,p_n$ respectively
and $\sum_{i=1}^n p_i = \mu$, where $\mu >0$ is a constant. Gleser \cite{Gleser75Schur}  proved the following:
suppose $\lfloor \mu - 2 \rfloor \le x \le \lceil \mu+2 \rceil$, then the tail probability $\Pr(\sum_{i=1}^n X_i \ge x)$ as
a function of $p_1,p_2, \cdots, p_n$ is Schur concave. This implies that the bound is maximized when $p_1 =
p_2 = \cdots = p_n = \mu/n$, providing a worst case bound.

The result can be extended to various cases. For instance, Merkle and Petrovi\'c extended this to the case
of independent geometrical and negative binomial random variables \cite{Merkle97Geom}. We believe
there is room for stronger results, in particular, if the results could be extended to log concave distributions. 

\section{Negative Association}

The concept of negatively associated random variables was first proposed by Joag-Dev \etal
\cite{JoagDev83NA}, which is one specific definition of negative dependence between 
random variables. The definition is as follows \cite[Definition 2.1]{JoagDev83NA}:
\begin{defn}
Random variables $X_1,X_2,\cdots, X_n$ are said to be {\em negatively associated} if for every pair of
disjoint subsets $\cA_1, \cA_2$ of $\{1,2,\cdots,n\}$,
\be
\Cov\Big( f(X_i, i \in \cA_1), g(X_j, j \in \cA_2) \Big) \le 0
\ee
whenever $f$ and $g$ are increasing.
\label{def:na}
\end{defn}

A major advantage of negative association over other definitions of negative dependence of random
variables is that increasing functions of disjoint sets of negatively associated random variables
are also negatively associated, \ie a closure property is satisfied.

Now, how do negatively associated random variables fit into designing randomized algorithms? 
The Chernoff bound \cite{MitzenmacherProb05} is a probability bound typically used to bound the error function of a
randomized algorithm. However, the chief assumption is that the collection of random variables
$X_1,X_2,\cdots,X_n$ must be independently distributed. This is generally not the case
in network algorithmics. 

Fortunately, the functions of interest on $X_1,X_2,\cdots,X_n$ are often
non-decreasing, so the Chernoff bound can be used as an upper bound on function of these
random variables, even though $X_1,X_2,\cdots,X_n$ are dependent. One such example is the sum
of the random variables, \ie
\ben
f(X_1,X_2,\cdots,X_n) = \sum_{i=1}^n X_i.
\een
Intuition-wise, it's because 
negative association ensures that the covariance between non-decreasing functions on subsets of
the random variables are non-positive. Compare this with independent random variables, where the
covariance would be exactly zero. Since the Chernoff bound is a moment bound, we know that 
non-positive covariance can only decrease variance, so the upper bound must be that of 
independent random variables. The discussion in the next section and Theorem \autoref{thm:na_sm}
will make this clearer.

At present, there is no well-developed framework, so proving that a set of random variables are
negatively associated is, aside from some special cases, a difficult task. 

One useful tool is the Fortuin-Kasteleyn-Ginibre (FKG) inequality \cite{Fortuin71FKG}. In
\cite{Dubhashi96FKGNA}, Dubhashi \etal showed how the FKG inequality can be used to prove
the negative associativity of some well-known distributions, such as negatively correlated
(binary) coins and the permutation distribution. The proofs, however, exploited clever 
arrangements of the events of a random variable on lattices \cite{Dubhashi96FKGNA}.
It must also be mentioned that this proof technique requires that the set of random variables
be discrete random variables.

Unfortunately, at present, it is often difficult to prove negative association despite its wide
implications. Though negative association is a very useful property to have, but because
proving this property is difficult, one way is to just show a weaker negative dependence,
such as, for a particular function $f$,
\ben
\E\left\lbrack \prod_{i=1}^m f(X_i) \right\rbrack 
\le  \prod_{i=1}^m \E\left\lbrack f(X_i) \right\rbrack .
\een
However, when negative association applies, we can elegantly sidestep more complicated
arguments based on martingales and the Azuma-Hoeffding inequality 
\cite{Azuma67Martingale,Hoeffding63Bounded}.

There has been results proving the negative association of random variables from 
well-known distributions.
Joag-Dev \etal \cite{JoagDev83NA} list some examples of negatively associated random variables:
\begin{enumerate}
\item \textit{Permutation distribution}: the joint distribution of a random vector $(X_1,X_2,\cdots,X_n)$,
$n > 1$, which takes as values all $n!$ permutations of $(x_1,x_2,\cdots, x_n)$, which is a set of 
$n$ real numbers, with equal probabilities, \ie $1/n!$. This covers two important cases:
\begin{itemize}
\item samples obtained from a finite population via random sampling without replacement, and
\item the joint distribution of ranks of a finite random sample from a population.
\end{itemize}
\item \textit{Several canonical multivariate distributions}: examples include the
\begin{itemize}
\item multinomial distribution,
\item multivariate hypergeometric distribution,
\item Dirichlet distribution,
\item Dirichlet compound multinomial distribution,
\item multinormal distributions having certain covariance matrices, and
\item negatively correlated normal random variables.
\end{itemize}
\item \textit{Marginal distributions of the row (column) vectors of a contingency table}: 
each cell count is an independent random sample taken from subpopulations that were
formed by partitioning the population according to the categories of the table.
\end{enumerate}

Aside from these examples, another is the famous \textit{balls-and-bins} model, where $n$ balls
are thrown uniformly at random into $m$ bins. If the balls are identical, then the joint distribution of
the number of balls in each bin $B_i$, $i=1,2\cdots,m$, once all the balls were thrown is then 
equal to the multinomial distribution. Thus, the balls-and-bins model is a generalization 
of the multinomial distribution, where one can have a probability $p_{i,k}$ to denote the
probability ball $k$ lands in bin $i$. The balls-and-bins model was indirectly listed in 
Joag-Dev \etal \cite{JoagDev83NA} as the ``convolution of unlike multinomial distributions'', but
no prove was given.

Dubhashi and Ranjan \cite{Dubhashi96BallsBins} showed that the random variables 
$B_i$, $i=1,2\cdots,m$, are in fact negatively associated. The intuition is clear, though the proof is
a little more complicated: since the number of balls $n$ is fixed, if a ball lands in bin $i$, then
there is one less ball that could possibly land in bin $j$. 

The balls-and-bins model arises in various problems in network algorithmics. For instance,
the power-of-two-choices heuristic used in load balancing and improving performance of
hash tables and Bloom filters is cast as a problem in distributing balls over bins. 




\subsection{Useful properties}

From the above, proving negative association can be challenging. 
In this section, we list some useful properties of negatively associated random variables.
These can be used to simplify proofs of the negative association of a set of random variables.

The following result was presented in \cite{JoagDev83NA} without proof. A proof is given here for
reference.
\begin{thm}
A pair of continuous random variables $X,Y$ are negatively associated if and only if 
\be
\Pr(X \le x, Y \le y) \le \Pr(X \le x) \Pr(Y\le y).
\label{eq:nqd}
\ee
\end{thm}
\begin{proof}
If $X,Y$ are negatively associated, then by choosing the indicator functions
$\mathbb{I} \{X \le x\}$ and $\mathbb{I}\{Y \le y\}$, we have
\begin{align*}
\Cov(\mathbb{I}\{X \le x\}, \mathbb{I}\{Y \le y\}) &\le 0\\
\E[\mathbb{I}\{X \le x\}\mathbb{I}\{Y \le y\}] - \E[\mathbb{I}\{X \le x\}] \E[\mathbb{I}\{Y \le y\}] &\le 0\\
\Pr(X \le x, Y \le y) - \Pr(X \le x) \Pr(Y\le y) &\le 0,
\end{align*}
and the result follows.

In the other direction, Hoeffding's identity \cite{Hoeffding63Bounded} implies
\ben
\Cov(f(X),g(Y)) = 2\int_{\Real} \int_{\Real} \Pr(f(X) \le u, g(Y) \le v) - \Pr(f(X) \le u)\Pr(g(Y) \le v) \,dx\,dy .
\een
However, applying the functions $f,g$ on \autoref{eq:nqd}, we get the inequality
\ben
\Pr(f(X) \le u, g(Y) \le v) - \Pr(f(X) \le u)\Pr(g(Y) \le v) \le 0.
\een
The result then follows.
\end{proof}

The result is also equivalent to the following: the pair $X,Y$ is negatively associated if and only if
\ben
\Pr(Y \le y\,|\,X\le x) \le \Pr(Y \le y) \text{ or } \Pr(X \le x\,|\,Y\le y) \le \Pr(X \le x).
\een
This equivalence provides much better intuition about negatively associated random variables. Additionally,
if the random variables satisfy \autoref{eq:nqd} then they are \textit{negative quadrature dependent}
\cite{Lehmann66NQD}.

Note that the theorem does not apply to discrete random variables \ie . However, in the case of
binary random variables, the variables are negatively associated if and only if 
they are negatively correlated \cite{Dubhashi96BallsNA}.

An important property for use together with Chernoff-type inequalities is the following: let 
$\cA_1,\cA_2, \cdots, \cA_m$ be disjoint subsets of the index set $\{1,2,\cdots,n\}$ and
$f_1,f_2,\cdots,f_m$ be increasing positive functions. Then if the set of random variables
$X_1,X_2,\cdots,X_n$ are negatively associated, it implies that
\ben
\E\left\lbrack \prod_{i=1}^m f_i(X_j, j\in \cA_i) \right\rbrack 
\le  \prod_{i=1}^m \E\left\lbrack f_i(X_j, j\in \cA_i) \right\rbrack .
\een
This means that the function $\prod_{i=1}^m f_i(X_j, j\in \cA_i)$ is log-concave. Moreover,
the it also follows that for $x_i \in \Real$, $i = 1,2,\cdots,n$, and $\cA_1, \cA_2$ are
disjoint subsets of $\{1,2,\cdots,n\}$, 
\begin{align}
\Pr(X_i \le x_i, i=1,2,\cdots,n) &\le \Pr(X_i \le x_i, i\in \cA_1) \Pr(X_j \le x_j, j\in \cA_2),\\
\Pr(X_i > x_i, i=1,2,\cdots,n) &\le \Pr(X_i > x_i, i\in \cA_1) \Pr(X_j > x_j, j\in \cA_2).
\end{align}

Several other properties listed in \cite{JoagDev83NA} are:
\begin{enumerate}
\item a subset of two or more negatively associated random variables are negatively associated,
\item a set of independent random variables are negatively associated,
\item increasing functions defined on disjoint subsets of a set of negatively associated random variables are
negatively associated,
\item the union of independent sets of negatively associated random variables are negatively associated.
\end{enumerate}



\section{Convex Ordering and Supermodularity}
\label{sec:supermodular}

We begin with the definition of (increasing) convex ordering:
\begin{defn}
Let $X$ and $Y$ be
random variables with finite means. Then $X$ is
less than $Y$ in (increasing) convex order, written $X \le_{cx} Y$
($X \le_{icx} Y$), if $\E[f(X)] \le \E[f(Y)]$ holds for all real
(increasing) convex functions $f$ such that the expectations
exist.
\label{def:convex_order}
\end{defn}

We can see how this ordering is useful: the moment generating function of any
random variable is an increasing convex function, provided that it exists. Then,
an ordering $X \le_{icx} Y$ implies that the moments of $X$ can be bounded by the
moments of $Y$, which would be useful when using Chernoff-type inequalities (tail
bounds).

We will also later on require the \textit{usual stochastic ordering}.
\begin{defn}
A random variable (or random vector) $X$ is stochastically less than or equal to a random 
variable (or random vector) $Y$, denoted as $X \le_{st} Y$, if and only if $\E[\phi(X)]
\le \E[\phi(Y)]$ for all increasing functions $\phi$ such that the expectations exist.
\label{defn:st_order}
\end{defn}

Supermodular functions (also know as $L$-superadditive functions \cite{Block89Ladditive}) are defined as follows:
\begin{defn}
A function $f: \Real^n \to \Real$ is called \textit{supermodular} if 
\ben
f(\bx \join \by) + f(\bx \meet \by) \ge f(\bx) + f(\by)
\een
where
\begin{align*}
\bx \join \by &:= \lbrack \max(x_1,y_1), \max(x_1,y_1), \cdots, \max(x_n,y_n)\rbrack,\\
\bx \meet \by &:= \lbrack \min(x_1,y_1), \min(x_1,y_1), \cdots, \min(x_n,y_n)\rbrack.
\end{align*}
A function $f$ is \textit{submodular} if and only if $-f$ is supermodular.
\end{defn}

Supermodular functions are often defined on lattices. Naturally, a partial order can be
defined on supermodular functions:
\begin{defn}
A random vector $\tbX = (X_1,X_2,\cdots,X_n)$ is said to be smaller than a random vector
$\tbY = (Y_1,Y_2,\cdots,Y_n)$ in the supermodular order, denoted by $\tbX \lsm \tbY$ if
$\E f(\tbX) \le \E f(\tbY)$ for all supermodular functions $f$ for which expectations exist. 
\end{defn}

A general result, by Christofides and Vaggelatou \cite[Theorem 1(b)]{Christofides04Supermod}, 
proved an ordering for all supermodular functions with negatively associated random variables. 
Supermodular ordering first appeared in \cite{Szekli94Queue}.
%Note that throughout the section, we assume expectations of the random variables exist.

\begin{thm}[Supermodular ordering and negative association]
Let $X_1,X_2,\cdots,X_n$ be a collection of negatively associated random variables and 
$Y_1,Y_2,\cdots,Y_n$ be independent random variables 
where each $Y_i$ possesses the same marginal distribution as $X_i$, $\forall i$. Then,
\ben
(X_1,X_2,\cdots,X_n) \lsm (Y_1,Y_2,\cdots,Y_n).
\een
\label{thm:na_sm}
\end{thm}

The fact that a composition between an increasing and convex function $f : \Real^n \to \Real$
and monotone and supermodular function $\phi: \Real^n \to \Real$ results in 
$f \circ \phi$ being supermodular results in the following corollary 
\cite[Corollary 1(b)]{Christofides04Supermod}:

\begin{cor}
Let $X_1,X_2,\cdots,X_n$ be a collection of negatively associated random variables and 
$Y_1,Y_2,\cdots,Y_n$ be independent random variables 
where each $Y_i$ possesses the same marginal distribution as $X_i$, $\forall i$. Then,
\ben
\phi(X_1,X_2,\cdots,X_n) \leicx \phi(Y_1,Y_2,\cdots,Y_n),
\een
for every $\phi$ monotone and supermodular.
\label{cor:na_icx}
\end{cor}



Let us consider a simple example. The linear function
\be
f(x) = \sum_i a_i x_i,
\label{eq:affine}
\ee
for all $a_i > 0$ is both increasing and supermodular (note that if there exists for
some $i$, $a_i < 0$, then the function is no longer increasing and supermodular). 
This function was encountered in the design of an SRAM-DRAM hybrid statistic counter
architecture for counting traffic flow sizes \cite{Zhao09DRAM}. 
We thus have a simpler alternative proof of \cite[Theorem 3]{Zhao09DRAM}:

\begin{thm}
Let $a_1,a_2,\cdots,a_n$ be constants, with $a_i > 0$, $\forall i$. Let $X_1,X_2,\cdots,X_n$ be real-valued
negatively associated random variables. Let $Y_1,Y_2,\cdots,Y_n$ be independent random variables 
where each $Y_i$ possesses the same marginal distribution as $X_i$, $\forall i$. Then,
\ben
\sum_{i=1}^n a_i X_i \lecx \sum_{i=1}^n a_i Y_i.
\een
\label{thm:na_convex}
\end{thm}
\begin{proof}
Note that $\E\lbrack \sum_{i} a_i X_i \rbrack = \E\lbrack \sum_{i} a_i Y_i \rbrack$. Then, convex
ordering is obtained instead of an increasing convex ordering. The result is just a consequence of
the monotone increasing nature and supermodularity of the function \autoref{eq:affine}.
\end{proof}

\begin{rem}
Note that it is not enough for $X_i$s to be negatively correlated. Theorem \autoref{thm:na_convex} asserts that
$X_i$s are negatively correlated for all convex functions $f$. This can be seen in the proof of \cite[Theorem 1(b)]{Christofides04Supermod},
where we need $\Cov(f'_+(X_1 + t), \mathbb{I}\{X_2 > t\}) \le 0$ to hold for all $t$.
\end{rem}

From this result, we immediately get the following as a corollary:
\begin{cor}
Assume the same notation as above. Let $X_1,X_2,\cdots,X_n$ be
a sample without replacement and $Y_1,Y_2,\cdots,Y_n$ be a sample with replacement from
a population $S = \{c_1,c_2,\cdots,c_N\}$, where $N > n$. Then,
\ben
\sum_{i=1}^n a_i X_i \lecx \sum_{i=1}^n a_i Y_i.
\een
\label{cor:sampling_wor_wr}
\end{cor}
\begin{proof}
It is well-known that since $X_1,X_2,\cdots,X_n$ are a set of samples without replacement, they 
are negatively associated \cite{JoagDev83NA}. 
The result follows from a direct application of Theorem \autoref{thm:na_convex}.
\end{proof}

Moreover,
Theorem \autoref{thm:na_convex} can be extended to show
\ben
\max_{1\le k \le n}\sum_{i=1}^k a_i X_i \leicx \max_{1\le k \le n} \sum_{i=1}^k a_i Y_i.
\een
Such a result can be used for the worse case tail analysis.
Note that the ordering becomes a convex ordering \ie
\ben
\max_{1\le k \le n}\sum_{i=1}^k a_i X_i \lecx \max_{1\le k \le n} \sum_{i=1}^k a_i Y_i,
\een
if $\E[X_i] = \E[Y_i]$ for all $i$.
Other examples of useful functions are the order statistic functions, for instance, the first and last
order statistics, which are supermodular and submodular respectively. 

Now, a set of random variables $X_1,X_2,\cdots, X_n$ are \textit{exchangeable} if their joint distribution
is invariant under permutation. Suppose $X_i$ and $Y_i$ are exchangeable for $i=1,2,\cdots,n$. 
Then the following holds

\begin{thm}
Let $a_1,a_2,\cdots,a_n$ be constants, with $a_i > 0$, $\forall i$. Let $X_1,X_2,\cdots,X_n$ be real-valued
exchangeable and negatively associated random variables. Let $Y_1,Y_2,\cdots,Y_n$ be 
real-valued exchangeable independent random variables where each $Y_i$ possesses the same marginal 
distribution as $X_i$, $\forall i$. Then, for every $\cS \subseteq \{1,2,\cdots,n\}$,
\ben
\sum_{i \in \cS} a_i X_i \leicx \sum_{i \in \cS} a_i Y_i
\een
and
\ben
\max_{1 \le |\cS| \le n}\sum_{i \in \cS} a_i X_i \leicx \max_{1 \le |\cS| \le n} \sum_{i \in \cS} a_i Y_i .
\een
\label{thm:na_exchangeable}
\end{thm}
\begin{proof}
We can construct any subset $\cS$ as follows. Suppose
$|\cS| = k$. Consider a permutation of $X_i$ and $Y_i$, $X_{\sigma(i)}$ and $Y_\sigma(i)$ respectively for 
$i=1,2,\cdots,n$. Then, indices belonging to $\cS$ can be chosen simply by choosing the appropriate
permutation map $\sigma$ and then sampling the first $k$ indices. 
Since the random variables are exchangeable, the joint distribution of these random variables remain
unchanged, so they remain negatively associated. The theorem then follows by applying 
Theorem \autoref{thm:na_convex}.
\end{proof}

Ideas from majorization can then be combined with supermodularity as well. Here, we present a 
result found in \cite{Zhao10GlobalIceberg} on the design of a distributed algorithm for detecting
global icebergs in networks.
\begin{thm}[\cite{Zhao10GlobalIceberg}]
Let $f$ be a convex function and $X_1,X_2,\cdots,X_n$ be non-negative valued 
exchangeable random variables. Then, for two real vectors $\ba$ and $\bb$ with the
relation $\ba \le_M \bb$ implies 
\ben
\sum_{i=1}^n f(a_i) X_i \le_{icx} \sum_{i=1}^n f(b_i) X_i.
\een
\label{thm:majorization_exchageable}
\end{thm}

\subsection{Proof recipe}

Suppose we know that a set of random variables $X_1,X_2,\cdots,X_n$ are negatively associated.
The general result of Christofides and Vaggelatou suggests a recipe on proving tail bounds on a
function of disjoint subsets of $X_i$s: simply focus on proving the 
supermodularity (submodularity) of the function $f$. 

To do so, we require a closely related concept called \textit{(strictly) increasing differences}. For any
function $f : \Real^n \to \Real$, let any pair of indices $i,j \in \{1,2,\cdots,n\}$, and any vector
\ben
\hat{x}_{i,j} = (x_1,\cdots,x_{i-1},x_{i+1},\cdots,x_{j-1},x_{j+1},\cdots,x_n) \in \Real^{n-2}
\een
that is an exclusion of entries $i,j$, and
\ben
f_{\hat{x}_{i,j}} (x'_i, x'_j) := f(x_1,\cdots,x_{i-1},x'_i,x_{i+1},\cdots,x_{j-1},x'_j,x_{j+1},\cdots,x_n).
\een
A function $f$ then has (strictly) increasing differences if for any pair of distinct indices $i,j$ and any
vector $\hat{x}_{i,j}$, with $x_i \le x'_i (x_i < x'_i)$ and $x_j \le x'_j (x_j < x'_j)$,
\ben
f_{\hat{x}_{i,j}} (x_i, x_j) - f_{\hat{x}_{i,j}} (x_i, x'_j) (<)  \le f_{\hat{x}_{i,j}} (x'_i, x_j) - f_{\hat{x}_{i,j}} (x'_i, x'_j).
\een
The definition can be extended to the integer domain as well. For instance, we can see that the maximum bin load in
the balls and bins model is an example of an increasing differences function.

Then, following result would be useful in proving supermodularity of a function:
\begin{thm}
A function $f : \Real^n \to \Real$ is (strictly) supermodular if and only if $f$ has (strictly) increasing
differences.
\label{thm:increasing_diff}
\end{thm}

Moreover, if the 
supermodular function is twice differentiable, then to prove supermodularity one needs to show
\ben
\frac{\partial^2 \phi(\bx)}{\partial x_i \partial x_j} \ge 0,
\een
for $\bx \in \Real^n$ and all $i \ne j$. 

Example supermodular and increasing functions:
\begin{itemize}
\item $p$-norms: $\|\bx\|_p = \Big( \sum_{i} x_i^{p} \Big)^{1/p}$,
\item first order statistic \cite{Block89Ladditive}: $f(\bx) = \max_{1\le i\le n} x_i$,
\item product function over $\Real^n \times \Real^n$: $f(\bx,\by) = \sum_{i=1}^n x_i y_i$,
\item Cobb Douglas function: $f(\bx) = \prod_{i=1}^n x_i^{\alpha_i}$ on the set
$\{\bx\,|\,\bx \succeq 0\}$ ($\succeq$ denotes element-wise non-negativity), 
for $\alpha_i \ge 0$,
\item minimization: if $f_i(z)$ is increasing on $\Real$ for $i=1,2,\cdots,n$ then 
\ben
f(\bx) = \min_{i \in \{1,2,\cdots,n\}} f_i(x_i)
\een 
is supermodular on $\Real^n$.
\end{itemize}


\section{Log-Concave Distributions}

Here, we briefly mention about log-concave distributions. A more complete survey 
of the topic is found in \cite{An95LogConcave} and \cite{Saumard14LogConcave}.

A continuous distribution is \textit{log-concave} if its domain is a convex set and its
probability density function $f : \Real^n \to [0,1]$ satisfies
\be
f(\lambda \bx + (1-\lambda) \by) \ge 	f(\bx)^{\lambda} f(\by)^{1-\lambda}
\label{eq:log_concave}
\ee
for all $\bx, \by \in \text{dom}\,f$ and $0 < \lambda < 1$. We can see this is
equivalent to 
\ben
\log f(\lambda \bx + (1-\lambda) \by) \ge 	\lambda \log f(\bx) + (1-\lambda) \log f(\by)
\een
if $f$ is strictly positive. Examples include well-known distributions such as 
the normal distribution, Wishart distribution, Dirichlet distribution and Beta distribution.

In network algorithmics, for counting applications, we frequently deal with integer-valued (discrete)
random variables. The definition can be extended for integer-valued random variables with modifications. In
the discrete case, log-concave distributions are defined as probability mass functions satisfying
\ben
p_x^2 \ge p_{x-1} p_{x+1}
\een
where $p_x := \Pr(X = x)$ for all $x \in \Integer$ \cite{Keilson71DiscreteUni}. In the case of 
\cite[Theorem 1]{Hua08BRICK,Hua08BRICKJournal}, it was
shown that the independent samples from the Binomial distribution is log-concave. Other examples include
\cite{An95LogConcave,Keilson71DiscreteUni}:
\begin{enumerate}
\item Bernoulli trial,
\item the binomial distribution,
\item the Poisson distribution,
\item the geometric distribution, and
\item the negative binomial distribution.
\end{enumerate}
These canonical distributions are often encountered in counting sketch applications.

Let $\Omega$ denote the support of the probability density function of a continuous random variable
$X$.
\begin{defn}
A probability density $f(x)$ is \textit{unimodal} if there exists a mode $m \in \Omega$ such that $f(x) \le f(y)$ for all
$x \le y \le m$ or for all $m \le y \le x$. $f(x)$ is \textit{strongly unimodal} if the convolution of $f$ with any
unimodal $g$ is unimodal.
\label{defn:unimodal}
\end{defn}
A surprising result by Ibragimov \cite{Ibragimov65LogUnimodal} is the following:
\begin{thm}
A random variable $X$ is distributed according to a log-concave distribution if and only if its
density function $f(x)$ is strongly unimodal.
\label{thm:log_concave_unimodal}
\end{thm}
\noindent
This is useful, because one can check if a distribution is strongly unimodal more readily than using the
direct definition of a log-concave distribution.

Though Ibragimov's result applies to continuous random variables, by applying the definition of
log-concavity for discrete random variables, the results can be extended to log-concave distributions
in the discrete case \cite[Theorem 3]{Keilson71DiscreteUni}.

Log-concave distributions have several desirable properties for applications to randomized algorithms:
\begin{enumerate}
\item the tail of a log-concave distribution is \textit{subexponential}, \ie it decays faster than the tail of an
exponential distribution,
\item the convolution of log-concave distributions is log-concave \ie the family is closed under convolution.
\end{enumerate}
The first property is clearly useful in bounding the worst-case probability of ``bad'' events. 

The second is useful in proving more complex distributions are log-concave. For instance, the sum of unlike Bernoulli
trials, \ie the distribution of Bernoulli trials, each with parameter $p_i$, $i = 1,2,\cdots,n$, has a 
distribution that is the convolution of the distribution of the trials. Since a Bernoulli trial has a log-concave
distribution, then the sum of unlike Bernoulli trials also has a log-concave distribution. The sum of 
random variables is often encountered in many problems in network algorithmics.

\subsection{Efron's monotonicity theorem}

The importance of log-concave distributions is chiefly due to \textit{Efron's monotonicity theorem}
\cite{Efron65Polya}. The theorem states the following:
\begin{thm}
Suppose that $f: \Real^n \to \Real$ where $f$ is coordinate-wise non-decreasing and let 
\be
g(z) := \E \left[ f(X_1,X_2,\cdots,X_n)\,\Big|\, \sum_{i=1}^n X_i = z\right],
\label{eq:conditional}
\ee
where $X_1,X_2,\cdots,X_n$ are independent and log-concave. Then $g$ is non-decreasing.
\label{thm:efron}
\end{thm}
Note that this result also holds for integer-valued (discrete) random variables with 
a log-concave distribution.

An interesting consequence of Theorem \ref{thm:efron} is that the joint conditional distribution of
$X_1,X_2,\cdots,X_n$ given $\sum_{i=1}^n X_i$ is \textit{negatively associated} 
(almost surely)~\cite[Theorem 2.8]{JoagDev83NA}. For instance, this means samples with replacement from a finite population, conditioned
on a total sampling budget are negatively associated. From the previous section, negative association implies some
useful properties that can be used to bound worse case events, for example, as in Theorem \ref{thm:na_exchangeable}.

Moreover, by the definition of the usual stochastic order, Efron's theorem implies
\be
\left(X_1,X_2,\cdots,X_n\,\Big|\,\sum_{i=1}^n X_i = s\right) \lst \left(X_1,X_2,\cdots,X_n\,\Big|\,\sum_{i=1}^n X_i = t\right) 
\label{eq:conditional_dominance}
\ee
for $t > s$ if $X_1,X_2,\cdots,X_n$ are independent samples from log-concave distributions.

One regularly encountered bound is a bound we term the \textit{times-2 bound}: 
for any coordinate-wise non-decreasing function $f$, and
two sets of random variables $X_1,X_2,\cdots,X_n$ and $Y_1,Y_2,\cdots,Y_n$,
\be
\E[f(X_1,X_2,\cdots,X_n)] \le 2\E[f(Y_1,Y_2,\cdots,Y_n)],
\label{eq:times_two}
\ee
where $Y_i$ has the same marginal distribution as $X_i$ for all $i$, yet $Y_i$s are independent.

For the balls-into-bins model with $m$ identical balls and $n$ bins, 
this was proven rigorously by Mitzenmacher and Upfal 
\cite{MitzenmacherProb05}: the $X_i$s denote the number of balls in a bin after all the
balls have been thrown (which are clearly dependent), while the $Y_i$s denote the 
independent Poisson random variables with rate $m/n$.

We present a generalized version of the bound that applies to log-concave distributions.
\begin{thm}[Times-2 Bound]
Let $Y_1,Y_2,\cdots,Y_n$ be a set of independent log-concavely distributed random 
variables, and let the support set of the distribution of the random variable $\sum_{i=1}^n Y_i$ be 
$\Omega$. Suppose
\begin{enumerate}
\item there exists a proper subset $\Omega' \subset \Omega$, which is a restricted
version of $\Omega$, starting from $\tau$ to the end of the maximum support (can be $\infty$), and
\item $\tau$ is less than the median of $\sum_{i=1}^n Y_i$.
\end{enumerate}
Let $X_1,X_2,\cdots,X_n$ be a set of dependent random variables such that
\ben
\mu(X_1,X_2,\cdots,X_n) = \mu\left(Y_1,Y_2,\cdots,Y_n\,\Big|\, \sum_{i=1}^n Y_i = \tau \right),
\een
where $\mu(Z)$ is the distribution of a random variable or vector $Z$.
Then, for any coordinate-wise non-decreasing function $f(x_1,x_2,\cdots,x_n)$,
\be
\E[f(X_1,X_2,\cdots,X_n)] \le 2 \E[f(Y_1,Y_2,\cdots,Y_n)] .
\label{eq:twice_bound}
\ee
\label{thm:two_times_lconcave}
\end{thm}
\begin{proof}
Here we derive the result for the case when both $X_i$ and $Y_i$ are discrete for all $i$.
The proof applies similarly when both $X_i$ and $Y_i$ are continuous with some modifications.
The proof follows an outline from the proof of \cite[Theorem 1]{Hua08BRICK} and
\cite[p.~121]{MitzenmacherProb05}. For 
any coordinate-wise non-decreasing function $f(x_1,x_2,\cdots,x_n)$, 
\begin{align}
\nonumber
 \E[f(Y_1,Y_2,\cdots,Y_n)] &= \sum_{\ell \in \Omega} \E\left[ f(Y_1,Y_2,\cdots,Y_n) \,\Big|\, \sum_{i=1}^n Y_i = \ell \right] \Pr\left( \sum_{i=1}^n
 Y_i = \ell \right)\\
 \nonumber
 &\ge \sum_{\ell \in \Omega'} \E\left[ f(Y_1,Y_2,\cdots,Y_n) \,\Big|\, \sum_{i=1}^n Y_i = \ell \right] \Pr\left( \sum_{i=1}^n
 Y_i = \ell \right)\\
 \label{eq:efron_lower}
 &\ge \sum_{\ell \in \Omega'} \E\left[ f(Y_1,Y_2,\cdots,Y_n) \,\Big|\, \sum_{i=1}^n Y_i = \tau \right] \Pr\left( \sum_{i=1}^n
 Y_i = \ell \right)\\
 \nonumber
 &= \E\left[ f(Y_1,Y_2,\cdots,Y_n) \,\Big|\, \sum_{i=1}^n Y_i = \tau \right] \sum_{\ell \in \Omega'} \Pr\left( \sum_{i=1}^n
 Y_i = \ell \right)\\
 \nonumber
 &= \E\left[ f(X_1,X_2,\cdots,X_n) \right] \Pr\left( \sum_{i=1}^n Y_i \ge \tau \right)\\
 \label{eq:50th_pc}
 &\ge \frac{1}{2} \E\left[ f(X_1,X_2,\cdots,X_n) \right].
\end{align}
Inequality \autoref{eq:efron_lower} follows from the implication of Efron's theorem, that is \autoref{eq:conditional_dominance}
and Condition (i). Inequality \autoref{eq:50th_pc} follows since Condition (ii) implies
\ben
\Pr\left( \sum_{i=1}^n Y_i \ge \tau \right) \ge \frac{1}{2}.
\een
\end{proof}

\begin{rem}
Now if $X_1,X_2,\cdots,X_n$ has a joint distribution such that
\ben
\mu(X_1,X_2,\cdots,X_n) = \mu\left(Y_1,Y_2,\cdots,Y_n\,\Big|\, \sum_{i=1}^n Y_i = \tau \right)
\een
and since $Y_1,Y_2,\cdots,Y_n$ given $\sum_{i=1}^n Y_i = \tau$ are negatively associated 
(almost surely), by implication, $X_1,X_2,\cdots,X_n$ are negatively associated (almost surely).
The proof, which we omit here, is via contradiction.
\label{rem:na}
\end{rem}

We can say a little more about the result. In Condition (ii), it is stated that $\tau$ is less than the
median of $\sum_{i=1}^n Y_i$. Often, what one would like to do is to set $\tau$ to be the mean
of $\sum_{i=1}^n Y_i$. We know that log-concave distributions are strongly unimodal,
so the distribution of $\sum_{i=1}^n Y_i$ is strongly unimodal. For 
the condition to be satisfied, we require that the mean must be less than or equal to the median.

The result has interesting implications: we can use a log-concavely distributed random variables 
to bound a set of (negatively) dependent random variables. In the balls-and-bins model, the chosen distribution for the 
$Y_i$s are the Poisson \cite{MitzenmacherProb05} and Binomial \cite{Hua08BRICK} random variables. Our result
shows that a wider class of distributions can be applied, so as long as the conditions of Theorem 
\ref{thm:two_times_lconcave} are satisfied.

With this result, we can now derive simple bounds. One trick is to observe that the indicator 
function is a coordinate-wise non-decreasing function, so we can set 
\ben
f(x_1,x_2,\cdots,x_n) = \mathbb{I}\Big\{ g(x_1,x_2,\cdots,x_n) > c \Big\},
\een
for some constant $c$ and some function $g$. Then, assuming the conditions of Theorem 
\ref{thm:two_times_lconcave} are satisfied, this gives us
\begin{align*}
%\E \mathbb{I}\Big\{ \mathbb{I}\{ g(X_1,X_2,\cdots,X_n) > c \} \Big\} &\le 2\E \mathbb{I}\Big\{ \mathbb{I}\{ g(Y_1,Y_2,\cdots,Y_n) > c \} \Big\}\\
\Pr\Big( g(X_1,X_2,\cdots,X_n) > c \Big) &\le 2\Pr\Big(  g(Y_1,Y_2,\cdots,Y_n) > c \Big).
\end{align*}
Since a tail bound is more readily derived for the $Y_i$s since they are independent, we know that the bound on 
$g(X_1,X_2,\cdots,X_n)$ is at most twice the tail bound probability on the $Y_i$s. 

As an example, in the balls-and-bins model, we can set
\ben
g(x_1,x_2,\cdots,x_n) = \max_{1 \le i \le n} x_i,
\een
that is, the maximum load over all bins. We then choose $Y_i$s from the Poisson distribution with rate
$\alpha = m/n$. Work by Kimber \cite{Kimber83PoissonMax} showed that 
\be
\Pr\left( \max_{1 \le i \le n} Y_i > \frac{\log n}{\log \log n} + 1\right) \to 0,
\ee
for fixed $\alpha$ as $n \to \infty$, so with high probability, the maximum load of a bin 
is $O(\log n/ \log \log n)$, confirming the results in literature. 

%For log-concave distributions, we can prove a stochastic ordering. We state the following result
%without proof, which can be found in \cite[Theorem 1.A.1]{Shaked07Sorders}:

%\begin{lem}
%Let $X$ and $Y$ be two random variables (or vectors). Then, $X \le_{st} Y$ if and only if there
%exists $X'$ and $Y'$ such that their measures $\mu(X') = \mu(X)$, $\mu(Y') = \mu(Y)$, and
%$\Pr(X' \le Y') = 1.
%\label{lem:st_order_equiv}
%\end{lem}


As an aside, there is no known result on the necessary condition for \autoref{eq:conditional} to be coordinate-wise
non-decreasing with respect to $z$. However, by following the proof of Efron's theorem, 
it is clear that the properties of the distribution must be preserved under convolution, since adding $X_{n+1}$ to 
a sequence $X_1,X_2,\cdots,X_n$ should preserve the joint distribution in order for the condition to hold.
For instance, log-concave distributions satisfy this condition because they are closed under convolution. 

\section{Conclusion}

Stochastic orders are excellent tools to aid in the design of randomized algorithms. We see this through their
application in simplifying proofs for bounding the errors of these algorithms. Though our focus is on 
network algorithmics, stochastic orders can be applied in a broader context. We hope that through our
paper, researchers are made aware of the benefits of exploiting these tools. 


\section*{Acknowledgment}

This work is supported in part by US NSF through grant CNS 1302197 and the Australian 
Research Council (ARC) grant DP110103505..

\bibliography{TuneBib}
\bibliographystyle{abbrv}
\end{document}